\documentclass{article}
\usepackage{biblatex}
\usepackage{hyperref}
\usepackage{authblk}

\bibliography{db}{}
\setlength{\emergencystretch}{3em}

\title{A parallel implementation of Strongly Connected Divide and Conquer.}
\author[1]{Adam Kurkiewicz\thanks{Adam@Kurkiewicz.pl}}
\author[1]{Iva Babukova\thanks{ibabukova@gmail.com}}
\author[2]{Huw Evans}
\affil[1]{School of Mathematics \& Statistics, School of Computing Science, The University of Glasgow}
\affil[2]{SAS Institute Inc}
\date{}

\begin{document}

\maketitle

\section{Abstract}

\par
This work presents a parallel Java implementation of the Strongly Connected Divide and Conquer (SCDC) algorithm \cite{fleischer00}, which is used to identify strongly connected components of a directed graph. Our main goal is to achieve a good level of performance and scalability. Both stages of the algorithm -- the search of random vertex descendants and predecessors, and subsequent recursive invocation -- are parallelised.

\par
Descendants and predecessors are explored using reachability analysis. The algorithm used is a parallel non-deterministic search, optimised to use a custom stack of vertices as opposed to a stack of recursive calls, which reduces its space complexity. The algorithm accepts as a parameter a granularity level, which is set experimentally to provide different scalability characteristics for multi-core machines.

\par
Recursive invocation is optimised to allow recursion only on sufficiently large subgraphs. For small subgraphs we are using Tarjan's \cite{tarjan72} algorithm, which we show gives performance gain. The size we consider sufficiently large is tuned experimentally.

\par
Performance of such optimised SCDC is compared with our implementations of single-threaded Tarjan's \cite{tarjan72} and Kosaraju-Sharir's \cite{sharir81} algorithms, and unoptimised multi-threaded SCDC \cite{fleischer00} on various classes of random and real world graphs. We discuss the impact of our optimisations and show the direction for further improvements.
\printbibliography

\end{document}

